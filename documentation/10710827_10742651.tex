\pagebreak

\hypertarget{reti-logiche---documentazione-prova-finale}{%
\section{Reti Logiche - Documentazione Prova
Finale}\label{reti-logiche---documentazione-prova-finale}}

Documentazione relativa al progetto finale del corso ``\textbf{Reti
Logiche}'', professore Fornaciari William.

Studenti \href{https://github.com/EdoardoDAlessio}{D'Alessio Edoardo}
(\emph{codice persona 10710827}) e
\href{https://github.com/SimoneDeidier}{Deidier Simone} (\emph{codice
persona 10742651}).\newline\newline\newline

\emph{Anno Accademico 2022/2023}

\pagebreak

\hypertarget{introduzione}{%
\subsection{Introduzione}\label{introduzione}}

\hypertarget{specifica}{%
\subsubsection{Specifica}\label{specifica}}

Il tema della Prova Finale consiste nell'ideazione, realizzazione
mediante linguaggio VHDL e relativo testing di un sistema di inoltro
dati provenienti da una memoria esterna verso una delle possibili
quattro uscite. I dati in uscita devono essere persistenti nel tempo e
devono cambiare solamente in caso di riscrittura. Più precisamente, il
componente riceve indicazioni circa una locazione di memoria ed il
canale di output da utilizzare mediante un ingresso seriale da un bit,
mentre le uscite del sistema forniscono tutti i bit della parola di
memoria in parallelo.

Il componente da sintetizzare presenta le seguenti porte per
interfacciarsi con il circuito ed i sistemi esterni:

\begin{itemize}
\tightlist
\item
  \textbf{CLOCK} (\texttt{i\_clk}) è il segnale di clock in ingresso.
\item
  \textbf{W} (\texttt{i\_w}) è il segnale di input dati; esso comprende
  due bit relativi all'uscita verso la quale indirizzare il dato di
  memoria
  (\(2\;\text{bit}\implies2^2=4\;\text{valori}\implies Z_0,Z_1,Z_2,Z_3\)),
  ed un numero \(0\leq x\leq 16\) di bit relativi all'indirizzo di
  memoria stesso. \emph{La memoria accetta solamente indirizzi 16 bit,
  pertanto il valore letto in input viene esteso con un numero di bit a
  0 tali per renderlo della dimensione desiderata}.
\item
  \textbf{START} (\texttt{i\_start}) è il segnale di in ingresso che,
  fintatntoché vale 1, segnala al componente di leggere il bit sulla
  linea \emph{W}. Il segnale è sicuramente attivo almeno per 2 cicli di
  clock (\emph{lettura dell'uscita sulla quale visualizzare il valore in
  output}) e per un massimo di 18 (\emph{2 cicli + 16 per la lettura
  dell'indirizzo di memoria}).
\item
  \textbf{RESET} (\texttt{i\_rst}) è il segnale che inizializza la
  macchina per ricevere il primo segnale di \emph{START}.
\item
  \textbf{Z\_0, Z\_1, Z\_2, Z\_3}
  (\texttt{o\_z0,\ o\_z1,\ o\_z2,\ o\_z3}) sono i quattro canali di
  output, ognuno da 8 bit ciascuno (\emph{i valori di output della
  memoria a cui il componente si interfaccia sono sempre di dimensione 8
  bit}).
\item
  \textbf{DONE} (\texttt{o\_done}) è il segnale di output che comunica
  la fine dell'elaborazione, esso \emph{assume il valore 1 nel ciclo di
  clock in cui il componente mostra i valori aggiornati su tutte e
  quattro le uscite}).
\item
  \textbf{MEMORY ADDRESS} è il segnale di output da 16 bit che manda
  l'indirizzo alla memoria.
\item
  \textbf{MEMORY DATA} è il segnale da 8 bit proveniente dalla memoria
  in seguito ad una richiesta di lettura.
\end{itemize}

\pagebreak

\begin{itemize}
\tightlist
\item
  \textbf{MEMORY ENABLE} (\texttt{o\_mem\_en}) è il segnale di
  \emph{ENABLE} da dover mandare alla memoria per poter comunicare sia
  in lettura che in scrittura.
\item
  \textbf{MEMORY WRITE ENABLE} (\texttt{o\_mem\_we}) è il segnale di
  \emph{WRITE ENABLE}; è necessario che questo segnale valga \emph{1}
  per poter scrivere in memoria, mentre per leggere esso deve essere
  \emph{0}.
\end{itemize}

\begin{figure}
\centering
\includegraphics{../../../../_resources/Screenshot 2023-08-27 alle 14.41.18.png}
\caption{\emph{Interfaccia del componente descritta in VHDL.}}
\end{figure}

\pagebreak

\hypertarget{esempio-di-funzionamento}{%
\subsubsection{Esempio di
funzionamento}\label{esempio-di-funzionamento}}

\begin{figure}
\centering
\includegraphics{../../../../_resources/Screenshot 2023-08-27 alle 14.42.05.png}
\caption{\emph{Esempio di funzionamento del componente da progettare.}}
\end{figure}

Nell'esempio il segnale di \emph{START} vale \emph{1} per 7 cicli di
clock: durante il primo ed il secondo ciclo di clock il componente legge
il valore del segnale input \emph{W} che andrà ad indicare su quale
uscita mostrare l'output
(\(\text{INPUT}=(01)_2\implies(1)_{10}\implies Z_1\)); durante i
successivi cinque cicli di clock in cui il segnale \emph{START} rimane
ad \emph{1}, il componente legge il valore di \emph{W}, il quale questa
volta indica l'indirizzo di memoria da cui andare a leggere il dato da
mostrare in output (\emph{il componente legge solamente 5 bit di
indirizzo, ovvero \((10110)_2\), ma alla memoria andrà a richiedere il
valore all'indirizzo \((0000000000010110)_2\), estendendo con tutti 0 il
valore letto in input per formare un indirizzo da 16 bit}). Il
componente interrogherà la memoria esterna richiedendo il valore (\(D\))
presente all'indirizzo specificato dal'input e, una volta che la memoria
ha correttamente comunicato l'indirizzo al componente, esso lo mostra
sull'output indicato ad inizio sequenza. Il valore \(D\) viene mostrato
in output per un solo ciclo di clock, dopodiché l'uscita torna a
mostrare \emph{0} come valore; inoltre, nello stesso ciclo di clock, il
segnale \emph{DONE} vale \emph{1}, e tutte le altre uscite
(\(Z_0,\,Z_2,\,Z_3\)) mostrano l'ultimo valore \(D_\text{old}\) che
hanno mostrato in output precedentemente (\emph{in questo esempio
mostrano tutti 0, si suppone che la sequenza sia avvenuta dopo aver dato
il segnale RESET al componente}).

\pagebreak

\hypertarget{architettura}{%
\subsection{Architettura}\label{architettura}}

\begin{figure}
\centering
\includegraphics{../../../../_resources/Screenshot 2023-08-31 alle 18.14.14.png}
\caption{\emph{Datapath del componente progettato.}}
\end{figure}

La figura mostra il \emph{datapath} intero del componente da noi
progettato (\emph{memoria esclusa, il componente è tutto ciò che sta
all'interno della linea tratteggiata grigia}) che rispetta le specifiche
richieste. Di segutio i componenti verranno analizzati più nel
dettaglio.

\hypertarget{fsm}{%
\subsubsection{\texorpdfstring{\emph{FSM}}{FSM}}\label{fsm}}

Questo componente implementa la \emph{macchina a stati finiti} che
abbiamo ideato per regolare la maggior parte dei segnali interni al
componente; essa presenta i seguenti stati:

\begin{itemize}
\tightlist
\item
  \textbf{S\_WAIT}, stato iniziale di \emph{wait} in cui la macchina
  aspetta l'inizio delle operazioni (\emph{ovvero il segnale}
  \texttt{start\ =\ 1}). Il comando \emph{reset} (\texttt{reset\ =\ 1})
  permette una corretta inizializzazione di tutti i registri ed i
  componenti. Una volta che arriva il segnale di \emph{start}, la
  macchina effettua la transizione verso lo stato
  \emph{SELECT\_OUTPUT\_LINE}.
\end{itemize}

\pagebreak

\begin{itemize}
\tightlist
\item
  \textbf{SELECT\_OUTPUT\_LINE}, stato in cui la macchina acquisisce il
  dato che indica su quale tra le quattro uscite visualizzare il nuovo
  valore richiesto dalla memoria; lo stato dura \emph{un solo ciclo di
  clock}, una volta terminato la macchina transisce verso lo stato
  \emph{TAKE\_MEM\_ADDR} (\emph{il perché della durata di un solo ciclo
  di clock per leggere due bit viene spiegato più approfonditamente
  nella sezione ``\textbf{Registri a scorrimento}''}).
\item
  \textbf{TAKE\_MEM\_ADDR}, stato in cui la macchina acquisisce i bit
  relativi all'indirizzo di memoria. Fintantoché il bit di \emph{start}
  vale \emph{1} la macchina rimane in questo stato, appena il bit
  \emph{start} vale \emph{0} avviene la transizione verso lo stato
  \emph{MEM\_REQ}.
\item
  \textbf{MEM\_REQ}, stato della durata di un singolo ciclo di clock nel
  quale la \emph{FSM} invia i dati alla memoria (\texttt{mem\_en\ =\ 1}
  \emph{ed il valore dell'indirizzo di memoria}). Lo stato successivo è
  \emph{MEM}.
\item
  \textbf{MEM}, stato della durata di un ciclo di clock nel quale la
  memoria accede all'indirizzo indicato dallo stato precedente. Alla
  fine di questo stato il valore del segnale \emph{mem\_data}
  proveniente dalla memoria conterrà il valore \(D\) di memoria
  all'indirizzo comunicato. Lo stato successivo è \emph{SET\_OUT\_REGS}.
\item
  \textbf{SET\_OUT\_REGS}, stato della durata di un ciclo di clock nel
  quale la \emph{FSM} invia ai registri di output il comando
  \emph{set\_output\_regs}, comando \textbf{master set} (\emph{descirtto
  più nel dettaglio nella sezione ``\textbf{Registri di output}''}).
  Durante questo stato il registro selezionato (\emph{stato
  SELECT\_OUTPUT\_LINE}) assume come valore il valore \(D\) proveniente
  dalla memoria (\emph{segnale mem\_data}). Lo stato successivo è
  \emph{SHOW\_OUTPUT}.
\item
  \textbf{SHOW\_OUTPUT}, stato della durata di un ciclo di clock nel
  quale la \emph{FSM} imposta il segnale \emph{show\_output} ad \emph{1}
  (\emph{descirtto più nel dettaglio nella sezione ``\textbf{Registri di
  output}''}), segnale che permette a tutti i registri di comunicare in
  output il valore da loro salvato. Lo stato successivo è
  \emph{S\_DONE}.
\item
  \textbf{S\_DONE}, stato della durata di un ciclo di clock nel quale la
  \emph{FSM} imposta il segnale \emph{done} ad \emph{1}. In questo ciclo
  di clock ogni uscita output, \(Z_0,\,Z_1,\,Z_2,\,Z_3,\) mostra il
  valore salvato nei registri ed il segnale \emph{o\_done} vale
  \emph{1}. A fine del ciclo la \emph{FSM} manda il segnale di
  \emph{reset} al registro da 16 bit per l'indirizzo di memoria, e
  ritorna allo stato \emph{S\_WAIT}.
\end{itemize}

\pagebreak

\begin{figure}
\centering
\includegraphics{../../../../_resources/Screenshot 2023-08-28 alle 16.47.33.jpg}
\caption{\emph{Diagramma degli stati della FSM.}}
\end{figure}

\pagebreak

\hypertarget{registri-a-scorrimento}{%
\subsubsection{Registri a scorrimento}\label{registri-a-scorrimento}}

Abbiamo utilizzato dei registri a scorrimento a sinistra (\emph{left
shift register}) per salvare le informazioni sia della scelta
dell'output, sia l'indirizzo di memoria; questa scelta si è rivelata
estremamente efficace dato che l'input viene dato serialmente un bit
alla volta, infatti utilizzare registri non a scorrimento avrebbe
implicato l'utilizzo di contatori accessori o di ulteriori accorgimenti
a nostro parere meno efficaci di un registro a scorrimento. Per quanto
riguarda il registro a scorrimento da 2 bit, esso salva nella posizione
dell'\emph{LSB} il valore in input (\emph{segnale W}) dopo aver
effettuato uno shift verso sinistra; questo accade solamente nello stato
\emph{SELECT\_OUTPUT\_LINE} dive il segnale di \emph{set}, ovvero
\emph{set\_two\_bit} vale \emph{1}. Dato che la \emph{FSM} impiega un
ciclo di clock per la transizione di stato, il segnale \emph{set} vale
\emph{1} anche nello stato \emph{S\_WAIT}, in modo tale che, appena il
segale \emph{start} vale \emph{1}, il primo bit del segnale \emph{W}
viene automaticamente letto e quindi è necessario una sola lettura
successiva per aver letto correttamente i due bit di scelta dell'output.
Per quanto riguarda il registro da 16 bit, il funzionamento è
essenzialmente identico al registro da 2 bit, ma con un accorgimento
fondamentale: all'inizio di ogni operazione, con il comando \emph{reset}
e \emph{rst\_addr}, il registro viene impostato con valore
``\(0000000000000000\)''; questo è estremamente comodo perché, essendo
un registro a scorrimento verso sinistra, se i bit inseriti per
l'indirizzo sono minori di 16, automaticamente il valore è esteso con
tutti \emph{0} come da specifica, senza dover effettuare ulteriori
operazioni. Etrambi i registri sono temporizzati in base al
\emph{clock}, ed anche il registro da 16 bit possiede un segnale di
\emph{set} chiamato \emph{set\_addr} che viene governato dalla
\emph{FSM} (\emph{impostato ad 1 nello stato TAKE\_MEM\_ADDR}); inoltre
il registro da 16 bit salva i valori in input solo se l'espressione
\(\text{SET\_ADDR}\land\text{START}\) risulta \emph{true}.

\hypertarget{demux}{%
\subsubsection{\texorpdfstring{\emph{DeMux}}{DeMux}}\label{demux}}

Il DeMultiplexer che abbiamo deciso di utilizzare nel nostro circuito ha
il compito di prendere in input il valore \(D\), che la memoria
restituisce dopo la richiesta da parte del componente, e di indirizzarlo
al \emph{registro di output} corretto in base all'uscita precedentemente
selezionata dall'input del sistema. Il segnale di input è quindi
\emph{mem\_data}, mentre il segnale di select del \emph{DeMux} è il
valore in memoria nel registro a 2 bit, ovvero
\emph{out\_two\_bit\_sreg}. La particolarità di questo componente è
l'output, infatti il segnale \emph{mem\_data} in input al \emph{DeMux} è
un segnale da 8 bit (\(D\)), invece ognuno dei quattro segnali di output
del \emph{DeMux}, ovvero i segnali \emph{input\_z\_x}, sono \textbf{9
bit}; abbiamo deciso infatti di aggiungere, oltre agli 8 bit del valore
di input, un segnale di \textbf{\emph{slave set}} che arrivi
direttamente al registro di output intreressato assieme al dato di
memoria da salvare. Questa scelta si è rivelata estremamente efficace
risparmiando informazioni da mandare alla \emph{FSM} (\emph{descirtto
più nel dettaglio nella sezione ``\textbf{Registri di output}''}).

\pagebreak

\begin{center}\rule{0.5\linewidth}{0.5pt}\end{center}

\hypertarget{esempio-di-funzionamento-demux}{%
\paragraph{\texorpdfstring{\emph{Esempio di funzionamento
DeMux}}{Esempio di funzionamento DeMux}}\label{esempio-di-funzionamento-demux}}

\emph{Supponendo il segnale di memoria \(D=(10100011)_2\) ed il segnale
di select del DeMux \(\text{out\_two\_bit\_sreg}=(01)_2\), il componente
manderà in output i seguenti valori:}

\[
\begin{cases}\text{out\_0}=(0)_2\\
\text{out\_1}=(10100011)_2\\
\text{out\_2}=(0)_2\\
\text{out\_3}=(0)_2\\
\text{set\_0}=(0)_2\\
\text{set\_1}=(1)_2\\
\text{set\_2}=(0)_2\\
\text{set\_3}=(0)_2\end{cases}
\]

\begin{center}\rule{0.5\linewidth}{0.5pt}\end{center}

\hypertarget{registri-di-output}{%
\subsubsection{Registri di output}\label{registri-di-output}}

Per permettere al nostro componente di mantenere in memoria per ogni
uscita l'ultimo valore \(D_\text{old}\) di memoria, abbiamo deciso di
implementare un \emph{registro di output} per ognuna delle quattro
uscite presenti. Questi sono registri da 8 bit non a scorrimento, che
presentano degli accorgimenti interessanti:

\begin{itemize}
\tightlist
\item
  Ogni registro presenta \emph{de segnali di set}, uno singolo per ogni
  registro che proviene dal \emph{DeMux}, chiamato anche
  \textbf{\emph{slave set}}, precedentemente descritto, mentre l'altro è
  in comune tra tutti i registri e proviene direttamente dalla
  \emph{FSM}, detto anche \textbf{\emph{master set}}. Un registro
  sovrascrive il dato in memoria con quello in ingresso
  \(\iff\text{SET}_\text{master}\land\text{SET}_\text{slave}=1\). Questa
  scelta di progettazione è risultata estremamente efficace perché ci ha
  permesso di poter scandire il momento del \emph{set} dei registri
  tramite \emph{FSM}, senza dover però far sapere alla \emph{FSM} a
  quale registro precisamente inviare il segnale di set.
\item
  Ogni registro presenta un segnale \emph{show\_output} regolato dalla
  \emph{FSM} ed in comune tra tutti i registri; questo segnale ci
  permette di scandire tramite \emph{FSM} il momento in cui i registri
  devono effettivamente mostrare in output il valore da loro salvato
  (\emph{ovvero nel ciclo di clock quando il segnale DONE vale 1}),
  mentre in tutti gli altri stati i registri porteranno in output un
  vettore da 8 bit di zeri, come richiesto da specifica.
\end{itemize}

Anche i \emph{registri di output} presentano una funzione di
\emph{reset} che azzera completamente il loro valore in memoria quando
l'omonimo comando viene inviato dalla \emph{FSM}.

\pagebreak

\hypertarget{risultati-sperimentali}{%
\subsection{Risultati sperimentali}\label{risultati-sperimentali}}

\hypertarget{sintesi}{%
\subsubsection{Sintesi}\label{sintesi}}

Il componente da noi ideato e descritto tramite linguaggio \emph{VHDL} è
risultato completamete e correttamente sintetizzabile. La sintesi ci ha
permesso inoltre di poter valutare alcuni aspetti cruciali del progetto
e del nostro componente, infatti, grazie al \emph{report sull'utilizzo},
abbiamo potuto appurare che il nostro componente utilizza solamente
\emph{registri flip-flop}, portando a zero il numero di \emph{registri
latch} utilizzati.

\begin{figure}
\centering
\includegraphics{../../../../_resources/Screenshot 2023-08-31 alle 12.42.29.png}
\caption{\emph{Parte del report di utilizzo di Vivado, come si può
notare i ``Register as Latch'' sono 0.}}
\end{figure}

Un altro elemento cruciale da verificare post sintesi è il
\emph{constraint} di tempo del clock, ovvero i ritardi che ogni porta e
componente logico apportano al sistema. Dal \emph{report sui tempi}
abbiamo potuto controllare che il \emph{constraint} di clock da
\emph{100 ns} è ampiamente soddisfatto, con uno \emph{\textbf{Slack}
pari a \textbf{97 ns}}.

\pagebreak

\begin{figure}
\centering
\includegraphics{../../../../_resources/Screenshot 2023-08-31 alle 16.46.56.png}
\caption{\emph{Calcolo dello Slack effettuato da Vivado in fase post
sintesi.}}
\end{figure}

\begin{figure}
\centering
\includegraphics{../../../../_resources/Screenshot 2023-08-31 alle 12.44.05.png}
\caption{\emph{Dal report di Vivado abbiamo potuto anche apprendere il
delay imposto dal nostro datapath: 2.387 ns.}}
\end{figure}

\hypertarget{simulazioni}{%
\subsubsection{Simulazioni}\label{simulazioni}}

Oltre alla corretta sintetizzazione del circuito, un altro degli
elementi fondamentali era la correttezza del componente. Per fare ciò ci
siamo appoggiati all'utilizzo di \emph{testbench}, ovvero codici
\emph{VHDL} che permettono di dare input ad un sistema o componente e
verificarne il corretto funzionamento tramite \emph{asserzioni} su
valori di segnali, registri o stati della \emph{FSM}. Dato che il
componente è stato completamente ideato da noi, abbiamo potuto applicare
alcune delle regole base del \textbf{\emph{white-box testing}}, andando
ad individuare la maggior parte dei casi e stati critici in cui il
nostro componente si può trovare; un rapido elenco dei casi testati:

\begin{itemize}
\tightlist
\item
  \emph{test di indirizzo vuoto}, abbiamo voluto testare il caso in cui
  il segnale \emph{start} rimane attivo per il minimo della sua durata
  (\emph{2 cicli di clock per l'uscita}), controllando così che il
  registro di memoria fosse correttamente inizializzato al valore
  \emph{0}.
\item
  \emph{test di indirizzo pieno}, contrariamente al test precedente,
  abbiamo voluto testare il caso in cui il segnale \emph{start} rimane
  attivo per il massimo possibile (\emph{18 cicli di clock, due per
  l'uscita, sedici per l'indirizzo}), dando un indirizzo casuale
  \(i\geq2^{15}=32768\), controllando che il \emph{MSB} dell'indirizzo
  fosse sempre \emph{1} e che il resto dei bit salvati nel registro
  fossero corretti.
\item
  \emph{test di massimo indirizzo}, a seguito del test precedente
  abbiamo voluto testare il caso in cui l'indirizzo di memoria sia
  \((1111111111111111)_2=2^{16}-1=(65535)_{10}\), controllando il
  corretto funzionamento del registro a scorrimento da 16 bit
  inizializzato a \emph{0}.
\end{itemize}

\pagebreak

\begin{itemize}
\tightlist
\item
  \emph{test di tutte le uscite}, abbiamo voluto testare una situazione
  in cui vengano dati quattro diversi indirizzi di memoria, ognuno
  assegnato ad una uscita diversa, controllando così il funzionamento
  generale del componente; abbiamo potuto controllare il corretto
  funzionamento dell'estensione a \emph{0} degli indirizzi non 16 bit,
  il funzionamento dei \emph{registri di output} che permettono di
  visualizzare sempre lo stesso valore \(D_\text{old}\) se non
  sovrascritti, ed il funzionamento del \emph{DeMux} per selezionare le
  quattro diverse uscite.
\item
  \emph{test di overwrite delle uscite}, dopo il test precedente abbiamo
  voluto controllare il funzionamento della sovrascrittura dei dati nei
  \emph{registri di output}, abbiamo così modificato la \emph{testbench}
  affinché vengano dati otto diversi indirizzi di memoria che vengono
  divisi equamente tra le quattro uscite, controllando che i dati in
  output alla fine di ogni operazione siano sempre corretti, anche dopo
  un \emph{overwrite} di dati di un \emph{registro di output}.
\item
  \emph{test generali sul comando reset}, abbiamo voluto controllare che
  il comando \emph{reset}, comando che in qualsiasi momento e stato il
  componente si trovi deve riportare tutto allo stato iniziale,
  funzionasse correttamente; per fare ciò abbiamo creato una
  \emph{testbench} che azionava il segnale \emph{reset} in momenti
  diversi: ad inizio test, dopo il segnale di \emph{start}, dopo che la
  memoria abbia comunicato il valore \(D\), dopo che il componente abbia
  salvato nel \emph{registro di output} il valore \(D\) da memoria, dopo
  che il componente abbia visualizzato sull'uscita il valore \(D\)
  salvato nel \emph{registro di output}. In ognuno di questi casi la
  \emph{testbench} dopo aver dato il segnale di \emph{reset} controllava
  che la \emph{FSM} si trovasse sempre nello stato \emph{S\_WAIT}, che
  ogni registro fosse reinizializzato a \emph{0}, e che ogni componente
  fosse pronto per ripartire con le operazioni nel modo corretto.
\end{itemize}

\pagebreak

\hypertarget{conclusioni}{%
\subsection{Conclusioni}\label{conclusioni}}

Ci riteniamo soddisfatti riguardo il componente da noi ideato,
sviluppato e testato. Siamo consapevoli che grazie alle conoscenze
apprese nel corso di \emph{Reti Logiche} siamo partiti da una buona idea
di circuito iniziale che, durante lo sviluppo, abbiamo modellato e
migliorato rendendolo sempre più efficace. La quantità di test e
simulazioni che sono state fatte ci permettono di essere consapevoli di
aver fatto un componente funzionante (\emph{anche nei casi limite}) e
che rispetti con grande margine tutti i \emph{constraint} che sono stati
imposti dalla specifica.

Questo progetto ci ha permesso di avere una visione molto dettagliata
della programmazione hardware, partendo dall'ideazione di un circuito e
dalla scelta dei componenti che si possono utilizzare, per poi arrivare
alla descrizione tramite linguaggio \emph{VHDL}, linguaggio
completamente nuovo per noi che nel corso di questo progetto abbiamo
saputo imparare ed apprezzare, così come il software \emph{Vivado}, che
ci ha permesso soprattutto di visualizzare graficamente ciò che accadeva
al nostro componente, aiutandoci moltissimo nella ricerca ed
individuazione dei bachi e delle criticità. La parte di ideazione e
stesura dei test ci ha molto messo alla prova, ma allo stesso tempo ci
ha consapevolizzato dell'importanza di questi ultimi per avere un
componente affidabile e funzionante.
